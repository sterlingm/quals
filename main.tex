
\documentclass[10pt,conference]{ieeeconf}
\begin{document}

\author{Sterling McLeod}
\title {Simultaneous Localization and Mapping (SLAM) Literature Survey}
\date {Month Day, Year}

\maketitle


\section {Problem Formulation}
    Simultaneous Localization and Mapping (SLAM) is the problem of determining a robot's pose and a map of its surrounding environment, given a set of percepts and robot trajectory. Mathematically, a robot's pose and map at time $t$ can be expressed as:
    
    \begin{equation}
    p(s^t,\theta | z^t, u^t, n^t)
    \end{equation}
    
    where $s^t$ is the robot's pose, $\theta$ is the environment map, $z^t$ is the latest sensor reading, $u^t$ is the latest control action, and $n^t$ is the set of landmarks observed at the current time.
    
    Landmarks are features of an environment that the robot uses to build a map. Some SLAM approaches represent a map simply as a set of landmark positions relative to the robot's pose. Other approaches use landmarks to build an occupancy-grid.
    
    SLAM has several important assumptions. Firstly, landmarks are assumed to be static. There can exist dynamic objects in an environment, but the features used to build a map should remain unchanged.

\section {Kalman Filters}
    

\subsection {Particle Filters}
    This is my second section's first subsection.

\section {Conclusion}
    This is my Conclusion.


\bibliography{biblio}
\bibliographystyle{unsrt}

\end{document}


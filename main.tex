\documentclass[10pt,conference]{ieeeconf}

\begin{document}


\nocite{PRM}


\author{Sterling McLeod}
\title {Real-time Robot Motion Planning: A Survey}
\date {2016}

\maketitle



\section{Problem Formulation}

	Robot Motion Planning is the problem of moving a robot from one position to another in the world. This problem is made difficult by several things: the kinematic structure of a robot, the complexity of its environment, how much information about the environment is known, the complexity of obstacles, and how much information about obstacles is known. 
	
	Real-time motion planning is the problem of performing motion planning on-line in order to react to a dynamic environment.
	
	Real-time motion planning involves several issues. 
	
	Most of the computational side

	
\section{Configuration Space}

A robot's \emph{configuration} is a vector specifying the values for each degree of freedom of the robot's kinematic structure. 

The \emph{Configuration Space} is an $n$ degree manifold mapping between a robot's transformations and Euclidean space. 




    

\section{Artificial Potential Fields}

In an artificial potential field approach, a robot is treated like a particle in a gradient field. The goal emits an attractive force that acts on the robot and the obstacles emit a repulsive force. The combination of forces results in a vector field. Figure~\ref{vector-field} illustrates an example.

At each location, a robot's motion is dictated by the magnitude and direction of the vector at the location.

\section{Cell Decomposition Methods}
	
	
\section{Sampling Based Methods}

A sampling based method was first proposed in 1996 \cite{PRM}. Sampling to build a probabilistic C-space was proposed due to the high complexity of building an exact C-space.

The basic Probabilistic Roadmaps algorithm works in two phases: learning and querying. 

The \emph{learning} phase builds a graph of configurations by sampling configurations.

Narrow passages are a limitation for this approach. In Figure \ref{narrow}(a), the free space between the two obstacles could be considered a narrow passage. With uniform sampling, there is little chance to that a point in the passage will be found. This may cause the planner  

The \emph{query} phase searches in the graph found during the learning phase to build a path from an initial configuration to a goal configuration. 

The PRM planner is considered \emph{probabilistically} complete. In most cases, it will not explore every point in the C-space. Given enough time, it will explore all points in the C-space and be able to determine if a path does not exist. 




\section{Collision Detection}
	
	Collision detection is the problem of determining whether one or more objects in a virtual environment are in contact. 
	
	Collision detection is the most computationally expensive task in robot motion planning. Thus, it is an imperative 



\section{Conclusion}
    This is my Conclusion.


\bibliography{rtrmp_bib}
\bibliographystyle{unsrt}

\end{document}


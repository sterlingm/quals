

\section{ Kinodynamic Motion Planning } \label{sec:kino}

Kinodynamic Planning is an approach to robot motion planning that integrates the kinematic and dynamic constraints of a robot into the planning aspect of the solution. In other words, it solves both path-planning and trajectory-following simultaneously. Prior motion planning approaches required these two problems to be done separately. Path-planning would be performed without considering a robot's dynamic constraints, and then a trajectory-following algorithm would interpolate points to move the robot along the path. The tuple describing a general motion planning problem from Section \ref{sec:probform} will contain two new values: $v_{max}$ and $a_{max}$, the maximum velocity and maximum acceleration, respectively, of the robot.


Since the goal of planning is usually to find an optimal solution (or the most-optimal available solution), it is possible that a kinodynamic planner may lead to an unsafe trajectory based on the robot's dynamic constraints. Therefore, a safety margin can be used to ensure that a solution maintains a certain distance from obstacles. A safety margin, $\delta$, is defined based on a robot's dynamic constraints. A $\delta_v$-\emph{safe kinodynamic solution} will move a robot from a starting state (position and velocity) to a goal state while avoiding obstacles by a safety margin $\delta_v$. Mathematically, the effect of the safety margin is the following:

\begin{equation}
\delta_v(c_0, c_1)(\dot{p}(t)) = c_0 + c_1||\dot{p}(t)||
\end{equation}

where $c_0$ and $c_1$ are positive scalar values.

An additional parameter to this problem is $l$, the side length of a cube that bounds all free condigurations. The complete tuple for a kinodynamic planning problem is $(A, W, Q, O, q_S, q_G, l, a_{max}, v_{max})$. The complete tuple for a $\delta_v$-\emph{safe kinodynamic planning} problem is $(A, W, Q, O, q_S, q_G, l, a_{max}, v_{max}, c_0, c_1)$. The values $a_{max}, v_{max}, l, c_0,$ and $c_1$ are \emph{kinodynamic bounds}.


The problem was first proposed in \cite{donald1993kinodynamic}, where an approximate solution was presented based on graph-search. The state space for this approach, denoted as $TC$, is the phase space of a robot's C-space. A point in this space represents both the robot's position and velocity, e.g. $S=(s, \dot{s})$. The state space is transformed into a directed graph where the vertices are discretized values from $TC$ and the edges represent trajectory segments.

The trajectory segments are formed from $(a, \tau)$-\emph{bangs}. Given a maximum acceleration, $a$, let the set of constant acceleration values $[-a,a]$ be denoted as $A$. A timestep $\tau$ can be chosen such that the velocity bound $v_{max}$ is an integer multiple of $a\tau$. A $(a, \tau)$-\emph{bang} is the result of applying an acceleration value from $A$ for duration $\tau$. 

The timestep $\tau$ is based on a function of $a_{max}$, $v_{max}$, an approximation parameter $\epsilon$, and scalar values $c_0$ and $c_1$. The scalar values define a safety margin and the approximation parameter defines a bound for closeness to the desired start and goal states. The chosen timestep is the largest value satisfying the following condition:

\begin{equation}
\tau \leq \frac{\epsilon}{13} min\left( \sqrt{ \frac{2c_0\epsilon}{a(c_1+1)}}, \frac{c_0\epsilon}{a(c_1+1)}\right)
\end{equation}

Given $\tau$, the initial state $S^*=(s, \dot{s})$ is chosen such that $s^* = s$ and $\dot{s}^*$ is the multiple of $a\tau$ closest to $\frac{\dot{s}}{1+\epsilon}$. 

Breadth-first Search is then applied to the embedded graph in $TC$. The initial state is $S^*$ and the goal state is any vertex within a certain threshold of the desired goal state, e.g. any vertex within $\left( \frac{a\tau^2}{2}, \frac{a\tau}{2} \right)$ of $\left(g, \frac{\dot{g}}{1+\epsilon}\right)$. Safety margin constraints are checked for each trajectory segment that the search algorithm considers.


A theorem is presented in this work to show that the complexity of the algorithm. Given the kinodynamic bounds of a problem, the upper bound for this algorithm's running time is


\begin{equation}
O\left( n \left[ \frac{lv\gamma^3}{\epsilon^6} \right] ^d \right)
\end{equation}

where $d=2,3$, $n$ is the number of bounding halfplanes on all obstacles, and 

\begin{equation}
\gamma = max \left( \frac{a_{max}(c_1+1)}{c_0}, \sqrt{ \frac{a_{max}(c_1+1)}{2c_0} }, \frac{a_{max}}{v_{max}}\right).
\end{equation}


This work can be used for kinodynamic planning of a rigid robot translating in a 3D space that contains convex polyhedra. This initial work is sufficient for problems with low degrees of freedom, but could not be shown to be a feasible approach to more complicated problems. Randomized approaches were proposed to handle kinodynamic planning problems with high degrees of freedom.

An RRT approach was proposed in \cite{lavalle2001randomized}. RRT lends itself to kinodynamic planning very easily because part of forming the connections in RRT is based on the control needed to move from node to node. The main difference is to change the state space from the robot's C-space to the phase-space of the C-space so that each state is a (\emph{position}, \emph{velocity}) pair.

Differential constraints, such as the non-holonomic constraint, are represented as the implicit function

\begin{equation}
\label{eq:rrt-kino} \dot{x} = f(x,u)
\end{equation}

where $u \in U$ and $U$ is the set of all inputs to a system. A numerical approximation for (\ref{eq:rrt-kino}) is needed to find the resulting state from applying $u$ to $x$. The work in \cite{lavalle2001randomized} uses fourth-order Runge-Kutta integration, but a similar numerical approximation technique will suffice.

The randomized kinodynamic planner based on RRT assumes that an environment is static. It extends the notion of an obstacle region in a state space to be a region of inevitable collision. This region contains all the states where the robot is either in collision or at a state where it will not be able to avoid an obstacle due to its dynamics. The region of inevitable collision subsumes the obstacle region in a state space.

This approach has been shown to work for several systems. For each system, the dynamic model must be defined as well as a distance metric in order to approximate a nearest neighbor for each node. The most complicated system described in experimental results is a three-dimensional spacecraft with a 12-dimensional state space. A bidirectional search method can also be applied to this work as in \cite{kuffner2000rrt}.

% Introduce NOT PRM approach, why needed?, the basics of the approach
The RRT approach did not consider dynamic obstacles, however. Dynamic obstacles in kinodynamic planning were considered in an PRM approach \cite{hsu2002randomized}. This work plans in the state$\times$time space of a robot by sampling to build an approximation of the search space. A vision module runs simultaneously with the planner to estimate obstacle motion.

% Define search space
Let $S$ denote the state space of a robot. $S$ can be the Configuration space of a robot, or it can include additional parameters. In the case of a car-like robot, a state $s \in S$ could be $(x,y,\theta)$ or it could also include velocity values, i.e. $(x,y,\theta,\dot{x},\dot{y},\dot{\theta})$. 

% Define query
Let $ST$ denote the state$\times$time space defined as $S\times[0,+\infty)$. Given an initial state$\times$time pair $(s_0, t_0)$ and a goal pair $(s_g, t_g)$, the output of this planner is a function $u:[t_0, t_g] \rightarrow \Omega$ that leads to a collision-free trajectory $\tau: t \in [t_0, t_g] \rightarrow \tau(t) = (s(t),t)$ such that $s(t_0)=s_0$ and $s(t_g)=s_g$.

% Describe algorithm
A query is solved by iteratively building a tree-shaped roadmap $T$ whose root is the initial state$\times$time pair through sampling until the goal is connected to the tree. Each iteration randomly selects an existing node on the roadmap, $(s,t)$, a time $t' \leq t_g$, and a control $u$. Then, the control input $u$ is applied to $(s,t)$ for time $t'$. This will form a trajectory in the state$\times$time space. If this trajectory lies in free space, then the endpoint $(s',t')$ is added as a node in the roadmap. The trajectory itself is stored as a directed edge between $(s,t)$ and $(s',t')$ and $u$ is also stored with the new edge. The algorithm exits when a trajectory leads to a new state that is in a neighborhood of the goal state. If some number of iterations $N$ occur and a solution has not been found, then the algorithm will return a failure to find a trajectory.

% Describe experiment setup (dimension of state spaces)
This approach was applied to two robot systems. The first system was a pair of non-holonomic bases that must remain within a minimum and maximum distance of each other. The state space of this system had 6 dimensions: $(x_1, y_1, \theta_1, x_2, y_2, \theta_2)$. The second system was an air-cushioned robot that moves frictionlessly on a flat table. The state space of this system had 4 dimensions: $(x, y, \dot{x}, \dot{y})$. When adding the time dimension to the state spaces, the algorithm built a roadmap in 7 and 5 dimensions, respectively.

In simulation, obstacle motion is given to the planner before executing the algorithm. The results verify that the planner is able to find a collision-free trajectory in the precense of moving obstacles for the majority of the time over the three cases shown. In experiments with real robots, obstacle motion is captured with an overhead camera. Obstacle motion is passed to the planner before executing and the planner assumes constant velocity for each obstacle. If the obstacle velocities significantly change, the camera will alert the planner with new information so that it can replan a new trajectory with new obstacle trajectories.

% Describe results
Several real robot experiments show that the planner was able to navigate the robot around moving obstacles when obstacle motion is known beforehand.


\section{ Kinodynamic Motion Planning } \label{sec:kino}

Kinodynamic Planning is an approach to robot motion planning that integrates the kinematic and dynamic constraints of a robot into the planning aspect of the solution. In other words, it solves both path-planning and trajectory-following simultaneously. Prior motion planning approaches required these two problems to be done separately. Path-planning would be performed without considering a robot's dynamic constraints, and then a trajectory-following algorithm would interpolate points to move the robot along the path. The tuple describing a general motion planning problem from Section \ref{sec:probform} will contain two new values: $v_{max}$ and $a_{max}$, the maximum velocity and maximum acceleration, respectively, of the robot.


The problem was first proposed in \cite{donald1993kinodynamic}, where an approximate solution was presented based on graph-search. The state space for this approach, denoted as $TC$, is the phase space of a robot's C-space. A point in this space represents both the robot's position and velocity, e.g. $S=(s, \dot{s})$. The state space is transformed into a directed graph where the vertices are discretized values from $TC$ and the edges represent trajectory segments.

The trajectory segments are formed from $(a, \tau)$-\emph{bangs}. Given a maximum acceleration, $a$, let the set of constant acceleration values be denoted as $A$. A timestep $\tau$ can be chosen such that the velocity bound $v_{max}$ is an integer multiple of $a\tau$. A $(a, \tau)$-\emph{bang} is the result of applying an acceleration value from $A$ for duration $\tau$.



Kinodynamic planning using PRM was proposed in \cite{hsu2002randomized}. The state space contains both configuration values and dynamic constraints.

The steps of the algorithm are nearly identical to Algorithm 1. However, the randomly sampled points are control values, rather than configurations.